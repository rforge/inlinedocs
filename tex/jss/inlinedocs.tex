\documentclass[article]{jss}

%%%%%%%%%%%%%%%%%%%%%%%%%%%%%%
%% declarations for jss.cls %%%%%%%%%%%%%%%%%%%%%%%%%%%%%%%%%%%%%%%%%%
%%%%%%%%%%%%%%%%%%%%%%%%%%%%%%

%% almost as usual
\newcommand{\inlinedocs}{\pkg{inlinedocs}}
\author{Toby Dylan Hocking\\INRIA Paris \And 
        Second Author\\Plus Affiliation}
\title{Sustainable, extensible documentation generation using \inlinedocs}

%% for pretty printing and a nice hypersummary also set:
\Plainauthor{Toby Dylan Hocking, Second Author} %% comma-separated
\Plaintitle{Sustainable, extensible documentation generation using
  \inlinedocs} %% without formatting
%\Shorttitle{A Capitalized Title} %% a short title (if necessary)
\newcommand{\R}{\proglang{R}}
%% an abstract and keywords
\Abstract{The concept of structured, interwoven code and documentation
  has existed for many years, but existing systems that implement this
  for the \R\ programming language do not take advantage of
  information present in the code. This article presents 2
  contributions for documentation generation the \R\ community. First,
  we propose a new syntax for inline documentation of \R\ code within
  comments adjacent to the relevant code, which allows for highly
  readable and maintainable code. Second, we propose an extensible
  system for parsing these comments, which allows the syntax to be
  easily augmented.}

\Keywords{Rd, documentation, documentation generation, literate
  programming, \proglang{R}}

\Plainkeywords{Rd, documentation, documentation generation, literate
  programming, R} %% without formatting
%% at least one keyword must be supplied

%% publication information
%% NOTE: Typically, this can be left commented and will be filled out by the technical editor
%% \Volume{13}
%% \Issue{9}
%% \Month{September}
%% \Year{2004}
%% \Submitdate{2004-09-29}
%% \Acceptdate{2004-09-29}

%% The address of (at least) one author should be given
%% in the following format:
\Address{
  Toby Dylan Hocking\\
  INRIA - Sierra project\\
  23, avenue d'Italie\\
  CS 81321 \\
  75214 Paris Cedex 13\\
  Telephone: +33/1 39 63 54 99\\
  E-mail: \email{Toby.Hocking@inria.fr}\\
  URL: \url{http://cbio.ensmp.fr/~thocking/}
}
%% It is also possible to add a telephone and fax number
%% before the e-mail in the following format:
%% Fax: +43/1/31336-734

%% for those who use Sweave please include the following line (with % symbols):
%% need no \usepackage{Sweave.sty}

%% end of declarations %%%%%%%%%%%%%%%%%%%%%%%%%%%%%%%%%%%%%%%%%%%%%%%


\begin{document}

%% include your article here, just as usual
%% Note that you should use the \pkg{}, \proglang{} and \code{} commands.

\section{Introduction}

\subsection[Existing documentation generation systems for R]{Existing
  documentation generation systems for \R}
package.skeleton

roxygen

Rdoc

\subsection{Documentation using inline comments}

motivation via function arguments, etc.

\section[The inlinedocs syntax]{The \inlinedocs\ syntax for inline
  documentation}
description, arguments, value using \verb+###+

examples using ex attr and structure(...,ex=function())

other items using \verb+##<<+

s4 classes, nested lists.

\section[The inlinedocs system of extensible documentation
generators]{The \inlinedocs\ system of extensible documentation
  generators}

\subsection{Structure of a Parser Function}

\subsection{forall/forfun}

\subsection{documentation lists}

\subsection{package.skeleton.dx}

how to call.

how to alter the default parsers: pkgdir/R/.inlinedocs.R variable
parsers or options("inlinedocs.parsers").

\section{Conclusions}

\end{document}
