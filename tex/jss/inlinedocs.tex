\documentclass[article]{jss}

%%%%%%%%%%%%%%%%%%%%%%%%%%%%%%
%% declarations for jss.cls %%%%%%%%%%%%%%%%%%%%%%%%%%%%%%%%%%%%%%%%%%
%%%%%%%%%%%%%%%%%%%%%%%%%%%%%%

%% almost as usual
\newcommand{\inlinedocs}{\pkg{inlinedocs}}
\author{Toby Dylan Hocking\\INRIA Paris \And 
        Second Author\\Plus Affiliation}
\title{Sustainable, extensible documentation generation using \inlinedocs}

%% for pretty printing and a nice hypersummary also set:
\Plainauthor{Toby Dylan Hocking, Second Author} %% comma-separated
\Plaintitle{Sustainable, extensible documentation generation using
  \inlinedocs} %% without formatting
%\Shorttitle{A Capitalized Title} %% a short title (if necessary)
\newcommand{\R}{\proglang{R}}
%% an abstract and keywords
\Abstract{The concept of structured, interwoven code and documentation
  has existed for many years, but existing systems that implement this
  for the \R\ programming language do not take advantage of
  information present in the code. This article presents 2
  contributions for documentation generation for the \R\ community. First, 
  we propose a new syntax for inline documentation of \R\ code within
  comments adjacent to the relevant code, which allows for highly
  readable and maintainable code. Second, we propose an extensible
  system for parsing these comments, which allows the syntax to be
  easily augmented.}

\Keywords{Rd, documentation, documentation generation, literate
  programming, \proglang{R}}

\Plainkeywords{Rd, documentation, documentation generation, literate
  programming, R} %% without formatting
%% at least one keyword must be supplied

%% publication information
%% NOTE: Typically, this can be left commented and will be filled out by the technical editor
%% \Volume{13}
%% \Issue{9}
%% \Month{September}
%% \Year{2004}
%% \Submitdate{2004-09-29}
%% \Acceptdate{2004-09-29}

%% The address of (at least) one author should be given
%% in the following format:
\Address{
  Toby Dylan Hocking\\
  INRIA - Sierra project\\
  23, avenue d'Italie\\
  CS 81321 \\
  75214 Paris Cedex 13\\
  Telephone: +33/1 39 63 54 99\\
  E-mail: \email{Toby.Hocking@inria.fr}\\
  URL: \url{http://cbio.ensmp.fr/~thocking/}
}
%% It is also possible to add a telephone and fax number
%% before the e-mail in the following format:
%% Fax: +43/1/31336-734

%% for those who use Sweave please include the following line (with % symbols):
%% need no \usepackage{Sweave.sty}

%% end of declarations %%%%%%%%%%%%%%%%%%%%%%%%%%%%%%%%%%%%%%%%%%%%%%%


\begin{document}

%% include your article here, just as usual
%% Note that you should use the \pkg{}, \proglang{} and \code{} commands.

\section{Introduction}

\subsection[Existing documentation generation systems for R]{Existing
  documentation generation systems for \R}
package.skeleton

roxygen

Rdoc

\subsection{Documentation using inline comments}

motivation via function arguments, etc.

\section[The inlinedocs syntax]{The \inlinedocs\ syntax for inline
  documentation}

Documentation of \R-objects (i.e. variables, functions, arguments,
list entries) are written inside special comments, that are directly adjacent
to the source code of the objects. Adjacent \inlinedocs-comments are

\begin{itemize}
\item \verb+##<<+ on the same line as the the \R-object
\item \verb+##+ on the lines directly following the \R-object
\item \verb+###+ on the line directly before the \R-object
\end{itemize}

Furthermore, \inlinedocs allows to put text of various documentation
sections anywhere in the function header or body of a function by using the
\verb+##SECTION<<+ notation.

The following subsections describe common usage of \inlinedocs-comments. 

\subsection{Documenting function signatures and return values}

The following examples demonstrates the basic documentation of a function.

%% twutz: who knows, how to deal with ##<< inside code environments?
% \begin{code}
% foo <- function(
% 	### Adding two arguments
% 	first           ##<< the first argument 
% 	,second=0       ##<< the second argument with a multi-line description
% 		## that is possibly better placed the details section  		
% ){
% 	first + second 
% 	### returns the sum of 
% 	### first and second
% }
% \end{code}

The comments are placed within sections of the corresponding 
documentation:
\begin{itemize}
\item Comment following the line of \code{function} goes to the description
section.
\item For each argument an item is created in the arguments section.
\item Comments following \code{return} or the last line of the body go to the
value section.
\end{itemize} 

Name, alias and title of the documentation are set to the function name. The
location of white spaces, brackets, default arguments and commas is quite flexible.

\subsection{Documenting descriptions, links, and references}
The comments of the following example are placed in the title, details, and
seealso sections respectively.
 
% \begin{code}
% bar <- function(
% 	### Demonstrating section comments
% 	##title<< bar-Function  		
% ){
% 	##details<<
% 	## This comment will show up in the details section.
% 	b <- rep(1,10)
% 	for( i in 2:10 ){
% 		##details<<
% 		## The cumulative sum, here, is calculated by a loop.\cr
% 		## Such comments can be placed anywhere in the code.
% 		b[i] <- b[i+1]
% 	}
% 	##seealso<< \code{\link{foo}}
% 	### vector of cumulative sume of 1:10
% }
% \end{code}


The \verb+##SECTION<<+ notation can be used for all documentation
sections (references, source, note, author, seealso, title, keyword, \ldots)
except for the examples section, which is handled in a special manner as shown below.

\subsection{Documenting list entries}

\subsection{Providing examples} 
examples using ex attr and structure(...,ex=function())

 
\subsection{Documenting S4 classes}

\section[The inlinedocs system of extensible documentation
generators]{The \inlinedocs\ system of extensible documentation
  generators}

\subsection{Structure of a Parser Function}

\subsection{forall/forfun}

\subsection{documentation lists}

\subsection{package.skeleton.dx}

how to call.

how to alter the default parsers: pkgdir/R/.inlinedocs.R variable
parsers or options("inlinedocs.parsers").

\section{Conclusions}

\end{document}
