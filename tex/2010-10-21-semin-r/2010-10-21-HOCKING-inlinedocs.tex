% -*- mode: noweb; noweb-default-code-mode: R-mode; -*-
\documentclass{beamer}
\usepackage{graphicx}
\usepackage{hyperref}

\title{
\url{http://inlinedocs.r-forge.r-project.org}}
\author{Toby Dylan Hocking \\ toby.hocking AT inria.fr}
\date{21 October 2010}

\AtBeginSection[]
{
  \begin{frame}<beamer>
    \frametitle{Outline}
    \tableofcontents[currentsection]
  \end{frame}
}

\AtBeginSubsection[]
{
  \begin{frame}<beamer>
    \frametitle{Outline}
    \tableofcontents[currentsection,currentsubsection]
  \end{frame}
}
\newcommand{\framet}[2]{\frame[containsverbatim]{
\begin{itemize}
\frametitle{#1}{#2}
\end{itemize}
}}

\begin{document}
\setkeys{Gin}{width=\textwidth}
\frame{\titlepage}

\section{General package structure}

\framet{Sharing your code with the R community}{
\item Most likely you have some interesting functions you would like
  to share.
\item You could just email your code.R to a colleague.
\item However, there is a standardized process for documenting,
  publishing and installing R code.
\item If you want your code to be used (and potentially modified) by
  anyone, then you should consider making a \textbf{package}.
p}

\framet{What is an R package?}{
\item It is a collection of code and data for a specific task, in a
  specific format.
\item Give your package a name, make a corresponding directory
  \texttt{pkgdir}
\item Required items:
  \begin{enumerate}
  \item pkgdir/R/*.R for R code.
  \item pkgdir/DESCRIPTION to describe its purpose, author, etc.
  \item pkgdir/man/*.Rd for \textbf{documentation}.
  \end{enumerate}
\item Optional items:
  \begin{itemize}
  \item pkgdir/data/* for data sets.
  \item pkgdir/src/* for C/FORTRAN/C++ source to be compiled
    and linked to R.
  \item pkgdir/inst/* for other files you want to install.
  \item pkgdir/po/* for international translations.
  \end{itemize}
\item All of these need to be in a standard format as described in
  ``Writing R Extensions'' in excrutiating detail.
} 

\framet{How to write the package?}{
\item Do it yourself! Read ``Writing R Extensions,'' only 141 pages in
  PDF form, as of 22 September 2010.
\item package.skeleton()
\item roxygen::roxygenize()
\item inlinedocs::package.skeleton.dx()
}

\framet{Documentation generation has several advantages}{
\item Documentation is written in comments, nearer to the source code
\item Can exploit the structure of the source code
\item Simplifies updating documentation
\item Reduces the probability of mismatch between code and docs
}

\section{roxygen: complete Rd support and call graphs}

\framet{Documentation in comments in a header}{
\item Comments start with \#'
}

\framet{Visualizing your functions using call graphs}{
\item @callGraph
}

\section{inlinedocs}

\section{Conclusions and references}

\framet{Documentation}{
\item To publish your package:
\begin{enumerate}
  \item Write your code in pkgdir/R/code.R
  \item Write a pkgdir/DESCRIPTION
  \item Write (or generate) documentation pkgdir/man/*.Rd
  \item R CMD check pkgdir
  \item R CMD build pkgdir
  \item Upload to ftp://cran.r-project.org/incoming user:anonymous
    password:your@email and email cran@r-project.org
\end{enumerate}
\item roxygen's strengths
\begin{itemize}
  \item Full support of Rd and namespaces
  \item Very nice call graphs
\end{itemize}
\item inlinedocs' strengths
\begin{itemize}
  \item Documentation closer to code
  \item Less repetition
\end{itemize}
}

\framet{References for learning more about package development}{
\item The definitive guide: help.start() then
  \href{http://cran.r-project.org/doc/manuals/R-exts.html}{Writing R
    Extensions},
\item The built-in package generator: ?package.skeleton
\item roxygen
  \begin{itemize}
  \item library(roxygen)
  \item ?roxygenize
  \item \url{http://roxygen.org/roxygen.pdf}
  \end{itemize}
\item inlinedocs
  \begin{itemize}
  \item library(inlinedocs)
  \item ?inlinedocs
  \item \url{http://inlinedocs.r-forge.r-project.com}
  \end{itemize}
\item Contact me directly: toby.hocking AT inria.fr,
  \url{http://cbio.ensmp.fr/~thocking/}
}

\end{document}

