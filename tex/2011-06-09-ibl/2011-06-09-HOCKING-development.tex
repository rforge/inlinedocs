% -*- mode: noweb; noweb-default-code-mode: R-mode; -*-
\documentclass{beamer}
\usepackage{graphicx}
\usepackage{hyperref}
\usepackage[all]{xy}

\title{Collaborative statistical software development using R-Forge\\
  \url{http://R-Forge.R-Project.org/}}
\author{Toby Dylan Hocking \\ toby.hocking AT inria.fr}
\date{9 June 2011}

\AtBeginSection[]
{
  \begin{frame}<beamer>
    \frametitle{Outline}
    \tableofcontents[currentsection]
  \end{frame}
}

\AtBeginSubsection[]
{
  \begin{frame}<beamer>
    \frametitle{Outline}
    \tableofcontents[currentsection,currentsubsection]
  \end{frame}
}
\newcommand{\framet}[2]{\frame[containsverbatim]{
\begin{itemize}
\frametitle{#1}{#2}
\end{itemize}
}}

\begin{document}
\setkeys{Gin}{width=\textwidth}
\frame{\titlepage}

\section{The R development model}

\begin{frame}
\frametitle{How do the authors of R write code?}
\begin{tabular}{ccc}
  Tool & R-core & R-Forge \\
  \hline
  mail & r-help@r-project.org & pkg-help@lists.r-forge.r-project.org\\
  public svn & http://svn.r-project.org/R/trunk & svn://svn.r-forge.r-project.org/svnroot/pkg \\
  dev svn & https://dev@svn.r-project.org/R/trunk & svn+ssh://dev@svn.r-forge.r-project.org/svnroot/pkg \\
  web & www.r-project.org & pkg.r-forge.r-project.org \\
  bugs & bugs.r-project.org & r-forge.r-project.org/tracker/?group\_id=496
\end{tabular}
\end{frame}

\framet{R-Forge is becoming the hub of R package development}{
  
}

\begin{frame}
  \frametitle{Collaboration level of projects}{
\end{frame}

\section{Using Subversion with R-Forge}

\section{Conclusions}

\framet{References for learning more about R-Forge}{
\item The definitive guide: help.start() then
  \href{http://cran.r-project.org/doc/manuals/R-exts.html}{Writing R
    Extensions}
\item The built-in package generator: ?package.skeleton
\item roxygen
  \begin{itemize}
  \item library(roxygen)
  \item ?roxygenize
  \item \url{http://roxygen.org/roxygen.pdf}
  \end{itemize}
\item R.oo:Rdoc
  \begin{itemize}
  \item library(R.oo)
  \item ?Rdoc (not very much documentation)
  \item \url{http://www.aroma-project.org/developers}
  \end{itemize}
\item inlinedocs
  \begin{itemize}
  \item library(inlinedocs)
  \item ?inlinedocs
  \item \url{http://inlinedocs.r-forge.r-project.org}
  \end{itemize}
\item Contact me directly: toby.hocking AT inria.fr,
  \url{http://cbio.ensmp.fr/~thocking/}
}

\end{document}

